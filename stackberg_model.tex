% !TEX TS-program = pdflatex
% !TEX encoding = UTF-8 Unicode

% This is a simple template for a LaTeX document using the "article" class.
% See "book", "report", "letter" for other types of document.

\documentclass[11pt]{article} % use larger type; default would be 10pt

\usepackage[utf8]{inputenc} % set input encoding (not needed with XeLaTeX)

%%% Examples of Article customizations
% These packages are optional, depending whether you want the features they provide.
% See the LaTeX Companion or other references for full information.

%%% PAGE DIMENSIONS
\usepackage{geometry} % to change the page dimensions
\geometry{a4paper} % or letterpaper (US) or a5paper or....
% \geometry{margin=2in} % for example, change the margins to 2 inches all round
% \geometry{landscape} % set up the page for landscape
%   read geometry.pdf for detailed page layout information

\usepackage{graphicx} % support the \includegraphics command and options

% \usepackage[parfill]{parskip} % Activate to begin paragraphs with an empty line rather than an indent

%%% PACKAGES
\usepackage{booktabs} % for much better looking tables
\usepackage{array} % for better arrays (eg matrices) in maths
\usepackage{paralist} % very flexible & customisable lists (eg. enumerate/itemize, etc.)
\usepackage{verbatim} % adds environment for commenting out blocks of text & for better verbatim
\usepackage{subfig} % make it possible to include more than one captioned figure/table in a single float
% These packages are all incorporated in the memoir class to one degree or another...

%%% HEADERS & FOOTERS
\usepackage{fancyhdr} % This should be set AFTER setting up the page geometry
\pagestyle{fancy} % options: empty , plain , fancy
\renewcommand{\headrulewidth}{0pt} % customise the layout...
\lhead{}\chead{}\rhead{}
\lfoot{}\cfoot{\thepage}\rfoot{}

%%% SECTION TITLE APPEARANCE
\usepackage{sectsty}
\allsectionsfont{\sffamily\mdseries\upshape} % (See the fntguide.pdf for font help)
% (This matches ConTeXt defaults)

%%% ToC (table of contents) APPEARANCE
\usepackage[nottoc,notlof,notlot]{tocbibind} % Put the bibliography in the ToC
\usepackage[titles,subfigure]{tocloft} % Alter the style of the Table of Contents
\renewcommand{\cftsecfont}{\rmfamily\mdseries\upshape}
\renewcommand{\cftsecpagefont}{\rmfamily\mdseries\upshape} % No bold!



\usepackage{amsmath}
\usepackage{amssymb}
\usepackage[english]{babel}
%%% END Article customizations

%%% The "real" document content comes below...

\title{Stackberg Model}
\author{Qiu Jiayu}
%\date{} % Activate to display a given date or no date (if empty),
         % otherwise the current date is printed 

\begin{document}
\maketitle

\section{model}
\subsection{Original Model}

\begin{align}  
    &min \quad  x_i^T Q_i x_i + c_i^T x_i \nonumber\\
    &s.t. \quad A_ix_i \leq d_i
\end{align}






\subsection{Lagrange Model}
\begin{align}  
    L =  x_i^T Q_i x_i + c_i^T x_i + u_i^T( A_ix_i - d_i)
\end{align}


\begin{align} 
    KKT\begin{cases} 
            \frac{\partial L}{\partial x_i} = 2 Q_i x_i + c_i^T + A_i^T u_i = 0 \\
            -u_i \leq 0 \\
            u_i - M_iI_i \leq 0\\
            A_ix_i  \leq d_i \\
            - A_ix_i + I_i M_i \leq -d_i + E_i M_i
    \end{cases}
\end{align} 








the original problem is translated into the following problem
\begin{align}  
    &min \quad 1^T y_i \nonumber\\
    &s.t. \quad B_i y_i \leq r_i
\end{align}
\begin{align}  
    \begin{bmatrix}
         2 Q_i & A_i^T & 0\\
         -2 Q_i & -A_i^T & 0\\
         A_i & 0 & 0\\
         -A_i & 0 & diag(M_i)\\
         0 & diag(E_i) & diag(-M_i)\\
         0 & diag(-E_i) & 0\\
    \end{bmatrix}
    \begin{bmatrix}
        x_i \\
        u_i \\
        I_i 
    \end{bmatrix} \leq
    \begin{bmatrix}
        -c_{i}\\
        c_{i}\\
        d_i\\
        -d_i + M_i\\
        0\\
        0\\
    \end{bmatrix} 
\end{align}
T hour model:
\begin{align} 
    \begin{bmatrix}
        X_1 & X_2 & \cdots  & X_T\\
    \end{bmatrix} 
    \begin{bmatrix}
        Q_1 & 0 & 0 & 0\\
        0 & Q_2 & 0 & 0\\
        0 & 0 & \ddots  & 0\\
        0 & 0 & 0 & Q_T
    \end{bmatrix} 
    \begin{bmatrix}
        X_1\\
        X_2\\
        \vdots \\
        X_T
    \end{bmatrix} +
    \begin{bmatrix}
        c_1 & c_2 & \cdots  & c_T\\
    \end{bmatrix}
    \begin{bmatrix}
        X_1\\
        X_2\\
        \vdots \\
        X_T
    \end{bmatrix}
\end{align}

\begin{align} 
    \begin{bmatrix}
        A_1 & 0 & 0 & 0\\
        0 & A_2 & 0 & 0\\
        0 & 0 & \ddots  & 0\\
        0 & 0 & 0 & A_T
    \end{bmatrix} 
    \begin{bmatrix}
        X_1\\
        X_2\\
        \vdots \\
        X_T
    \end{bmatrix} \leq
    \begin{bmatrix}
        d_1\\
        d_2\\
        \vdots \\
        d_T
    \end{bmatrix}
\end{align}

\begin{align} 
    \begin{bmatrix}
        B_1 & 0 & 0 & 0\\
        0 & B_2 & 0 & 0\\
        0 & 0 & \ddots  & 0\\
        0 & 0 & 0 & B_T
    \end{bmatrix} 
    \begin{bmatrix}
        Y_1\\
        Y_2\\
        \vdots \\
        Y_T
    \end{bmatrix} \leq
    \begin{bmatrix}
        r_1\\
        r_2\\
        \vdots \\
        r_T
    \end{bmatrix}
\end{align}



\subsection{The Leader Model}
\begin{align}  
    &min \quad f(w)\nonumber\\
    &s.t. \quad Mw \geq b
\end{align}
\begin{align}  
    w = \{z, y_{1}, \cdots y_{N}\}
\end{align}
\begin{align}  
    M = \begin{bmatrix}
        -C_{1} & B_{1} & \cdots & \cdots & \cdots\\
        -C_{2} &\vdots  & B_{2} & \cdots & \cdots\\
         \vdots  & \vdots  & \vdots & \ddots  & \cdots\\
        -C_{N} & \cdots  & \cdots & \cdots & B_{N}\\
    \end{bmatrix}
\end{align}
\begin{align}  
    b = \begin{bmatrix}
        b_{1}^0\\
        b_{2}^0\\
        \vdots \\
        b_{n}^0\\
    \end{bmatrix}
\end{align}






\subsection{Example}
the model for each player  is
\begin{align}  
    &min \quad \frac{1}{2}bE_{si}^2 -aE_{si} + C_{si}P_{si}  \nonumber\\
    &s.t \begin{cases}
        E_{si} - E_{si}^{max} \leq 0\\
         -E_{si} + E_{si}^{min}\leq 0\\
         -P_{si}\leq 0\\
          P_{si} - P_{si}^{max} \leq 0
        \end{cases}
\end{align}
the model can be written in the form of matrix
\begin{align} 
    min \quad
    \begin{bmatrix}
        E_{ei}
        E_{si}
        P_{ei}
        P_{si}
    \end{bmatrix} 
    \begin{bmatrix}
        0 & 0 & 0 & 0\\
        0 & \frac{b}{2}  & 0 & 0\\
        0 & 0 & 0 & 0\\
        0 & 0 & 0 & 0
    \end{bmatrix} 
    \begin{bmatrix}
        E_{ei}\\
        E_{si}\\
        P_{ei}\\
        P_{si}
    \end{bmatrix} +
    \begin{bmatrix}
        0 & -a & 0 & C_{si}\\
    \end{bmatrix} 
    \begin{bmatrix}
        E_{ei}\\
        E_{si}\\
        P_{ei}\\
        P_{si}
    \end{bmatrix}
\end{align}

\begin{align} 
    \begin{bmatrix}
        0 & 1 & 0 & 0\\
        0 & -1 & 0 & 0\\
        0 & 0 & 0 & -1\\
        0 & 0 & 0 & 1
    \end{bmatrix} 
    \begin{bmatrix}
        E_{ei}\\
        E_{si}\\
        P_{ei}\\
        P_{si}
    \end{bmatrix} \leq
    \begin{bmatrix}
        E_{si}^{max}\\
        E_{si}^{min}\\
        0\\
        P_{s}^{max}
    \end{bmatrix}
\end{align}









\begin{align} 
    L = \frac{1}{2}bE_{si}^2 -aE_{si} + C_{si}P_{si} + u_1(E_{si} - E_{si}^{max}) + u_2(-E_{si} + E_{si}^{min}) +
     u_3(-P_{si}) + u_4(P_{si} - P_{si}^{max})
\end{align}
\begin{align}  
    \frac{\partial L}{\partial E_{si}} = bE_{si} - a + u_1 - u_2= 0 
\end{align}
\begin{align}  
    \frac{\partial L}{\partial P_{si}} = C_{si} - u_3 + u_4= 0 
\end{align}
\begin{align}  
    \begin{bmatrix}
         \begin{bmatrix}
            0 & 0 & 0 & 0\\
            0 & b  & 0 & 0\\
            0 & 0 & 0 & 0\\
            0 & 0 & 0 & 0
        \end{bmatrix}  & 
        \begin{bmatrix}
            0 & 0 & 0 & 0\\
            1 & -1 & 0 & 0\\
            0 & 0 & 0 & 0\\
            0 & 0 & -1 & 1
        \end{bmatrix}  &
        \begin{bmatrix}
            0 & 0 & 0 & 0\\
            0 & 0 & 0 & 0\\
            0 & 0 & 0 & 0\\
            0 & 0 & 0 & 0
        \end{bmatrix} \\
        \begin{bmatrix}
            0 & 0 & 0 & 0\\
            0 & -b  & 0 & 0\\
            0 & 0 & 0 & 0\\
            0 & 0 & 0 & 0
        \end{bmatrix}  & 
        \begin{bmatrix}
            0 & 0 & 0 & 0\\
            -1 & 1 & 0 & 0\\
            0 & 0 & 0 & 0\\
            0 & 0 & 1 & -1
        \end{bmatrix}  &
        \begin{bmatrix}
            0 & 0 & 0 & 0\\
            0 & 0 & 0 & 0\\
            0 & 0 & 0 & 0\\
            0 & 0 & 0 & 0
        \end{bmatrix} \\
        \begin{bmatrix}
            0 & 1 & 0 & 0\\
            0 & -1 & 0 & 0\\
            0 & 0 & 0 & -1\\
            0 & 0 & 0 & 1
        \end{bmatrix}  & 
        \begin{bmatrix}
            0 & 0 & 0 & 0\\
            0 & 0 & 0 & 0\\
            0 & 0 & 0 & 0\\
            0 & 0 & 0 & 0
        \end{bmatrix}   &
        \begin{bmatrix}
            0 & 0 & 0 & 0\\
            0 & 0 & 0 & 0\\
            0 & 0 & 0 & 0\\
            0 & 0 & 0 & 0
        \end{bmatrix} \\
        \begin{bmatrix}
            0 & -1 & 0 & 0\\
            0 & 1 & 0 & 0\\
            0 & 0 & 0 & 1\\
            0 & 0 & 0 & -1
        \end{bmatrix}  & 
        \begin{bmatrix}
            0 & 0 & 0 & 0\\
            0 & 0 & 0 & 0\\
            0 & 0 & 0 & 0\\
            0 & 0 & 0 & 0
        \end{bmatrix}   &
        \begin{bmatrix}
            M & 0 & 0 & 0\\
            0 & M & 0 & 0\\
            0 & 0 & M & 0\\
            0 & 0 & 0 & M
        \end{bmatrix} \\
        \begin{bmatrix}
            0 & 0 & 0 & 0\\
            0 & 0 & 0 & 0\\
            0 & 0 & 0 & 0\\
            0 & 0 & 0 & 0
        \end{bmatrix}  & 
        \begin{bmatrix}
            1 & 0 & 0 & 0\\
            0 & 1 & 0 & 0\\
            0 & 0 & 1 & 0\\
            0 & 0 & 0 & 1
        \end{bmatrix}   &
        \begin{bmatrix}
            -M & 0 & 0 & 0\\
            0 & -M & 0 & 0\\
            0 & 0 & -M & 0\\
            0 & 0 & 0 & -M
        \end{bmatrix} \\
        \begin{bmatrix}
            0 & 0 & 0 & 0\\
            0 & 0 & 0 & 0\\
            0 & 0 & 0 & 0\\
            0 & 0 & 0 & 0
        \end{bmatrix}  & 
        \begin{bmatrix}
            -1 & 0 & 0 & 0\\
            0 & -1 & 0 & 0\\
            0 & 0 & -1 & 0\\
            0 & 0 & 0 & -1
        \end{bmatrix}   &
        \begin{bmatrix}
            0 & 0 & 0 & 0\\
            0 & 0 & 0 & 0\\
            0 & 0 & 0 & 0\\
            0 & 0 & 0 & 0
        \end{bmatrix} \\
    \end{bmatrix}
    \begin{bmatrix}
        E_{ei} \\
        E_{si} \\
        P_{ei} \\
        P_{si} \\
        u_1\\
        u_2\\
        u_3\\
        u_4\\
        I1\\
        I2\\
        I3\\
        I4 
    \end{bmatrix} \leq
    \begin{bmatrix}
        \begin{bmatrix}
            0\\
            a\\
            0\\
            -C_{si}
        \end{bmatrix}\\
        \begin{bmatrix}
            0\\
            -a\\
            0\\
            C_{si}
        \end{bmatrix}\\
        \begin{bmatrix}
            E_{si}^{max}\\
            E_{si}^{min}\\
            0\\
            P_{s}^{max}
        \end{bmatrix}\\
        \begin{bmatrix}
            -E_{si}^{max} + M\\
            -E_{si}^{min} + M\\
            M\\
            -P_{s}^{max} + M
        \end{bmatrix}\\
        \begin{bmatrix}
            0\\
            0\\
            0\\
            0
        \end{bmatrix}\\
        \begin{bmatrix}
            0\\
            0\\
            0\\
            0
        \end{bmatrix}\\
    \end{bmatrix} 
\end{align}

































\end{document}



